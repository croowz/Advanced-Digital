\documentclass[12pt]{article}
\usepackage{setspace}
\setlength{\parindent}{4em}
\usepackage{fancyvrb}
\usepackage{graphicx}
\usepackage{geometry}
\renewcommand\thesection{\arabic{section}}
\renewcommand\thesubsection{\thesection.\arabic{subsection}}
\geometry{letterpaper, portrait, margin=1in}

%%%Title Page%%%
\title{\vspace{3cm}Lab 06\bigbreak Adding Memory and Bootstrapping Circuit}
\author{
{\normalsize
\begin{tabular}{l r r}
 & \textbf{Ryan Cruz} & \textbf{Zachary Davis}\\
\textbf{Category} & ryan.cruz25@uga.edu & zachdav@uga.edu\\
\hline
Pre-lab 						  & 50 & 50\\
In-lab Module \& Testbench Design & 50 & 50\\
In-lab Testbench Sim. \& Analysis & 50 & 50\\
In-lab FPGA Synthesis \& Analysis & 50 & 50\\
Lab Report Writing 				  & 50 & 50\\
\end{tabular}
}}
%%%%%%%%%%%%%%%%%

\begin{document}
\maketitle
\newpage
\setstretch{2.5} % for custom spacing
\tableofcontents
\setstretch{1} % for custom spacing
\newpage

\section{Lab Purpose} \vspace{-.7cm} \line(1,0){470}
	\paragraph{}
		Entering this lab, we now have all of the components to complete the design of the Toy Processor. The two main modules that have resulted from our previous labs are the datapath and the controller. In this lab, we will connect those two components and then test its behavior with a simple program by encoding certain instructions and executing them in a test fixture to view data inputs and their respective outputs. 
		
\section{Implementation Details} \vspace{-.7cm} \line(1,0){470}
		\subsection{Prelab}
			\hfill

		\begin{figure}[h]
		\centering
			\includegraphics[scale=.8]{Prelab.pdf}
			\caption{Memory Bootstrapping circuit to copy the entire contents of the ROM array into the RAM array.}
		\end{figure}
			
		\begin{figure}[h]
			%\includegraphics[scale=.6]{Prelab2.}
			\caption{Modification of the controller so that we do not transition from S0 to S1 before the memory bootstrapping is complete.}
		\end{figure}
		
		\newpage
	\subsection{Part 1 - Include ROM}
	The ROM array was given to us as a schematic of ROM32X1 modules. Below is a testbench to make sure it is operating correctly. We also initialized the first row of modules with the following values:

\begin{flushleft}
\begin{tabular}{|l|}
\hline
00000912 \\
00000910 \\
00000912 \\
00000990 \\
00000D12 \\
00000B14 \\
00000932 \\
00000149 \\
\hline
\end{tabular}
\end{flushleft}

		
		\begin{Verbatim}[frame=single, fontsize= \small]
//ROM Testbench
`timescale 1ns / 1ps

module ROM_array_ROM_array_sch_tb();

// Inputs
   reg [7:0] ADDR = 8'b00000000;

// Output
   wire [7:0] DATA_OUT;

// Bidirs

// Instantiate the UUT
   ROM_array UUT (
		.ADDR(ADDR), 
		.DATA_OUT(DATA_OUT)
   );
// Initialize Inputs
       initial begin
			#400;
			ADDR = 8'b00100000;
			#100;
			ADDR = 8'b00000001;
			end
endmodule

		\end{Verbatim}
		The test bench simply executes one address change to show that it is working properly. Refer to the Expiremental Results section to view the wavefrom results of this testbench.
		\newpage
		\subsection{Part 2 - RAM}
		The ROM array was given to us as a schematic as well. Below is a testbench to make sure it is operating correctly.

		\begin{Verbatim}[frame=single, fontsize= \small]
//RAM_array_tb.v
`timescale 1ns/1ps
module RAM_array_tb;

reg [7:0] ADDR = 8'b00000000;
reg CLK = 1'b0;
reg [7:0] DATA_IN = 8'b00000000;
reg WE = 1'b0;
wire [0:7] DATA_OUT1;

initial // Clock process for CLK
begin
	forever 
		begin
			CLK = 1'b0;
			#100;
			CLK = 1'b1; 
			#100;
		end
	end
RAM_array UUT (
.ADDR(ADDR),CLK(CLK),.DATA_IN(DATA_IN),.WE(WE),.DATA_OUT1(DATA_OUT1));

initial begin
#85;
DATA_IN = 8'b11111111;
#200;
WE = 1'b1;
ADDR = 8'b00010000;
#600;
WE = 1'b0;
#200;
ADDR = 8'b00100000;
#600;
ADDR = 8'b00010000;

end
endmodule
		\end{Verbatim}
		This test bench is similar to the ROM one in purpose, simply executing a write to make sure it works. Refer to the Expiremental Results section to view the wavefrom results of this testbench.

\newpage			
\section{Experimental Results}\vspace{-.7cm} \line(1,0){470}

% \begin{center}
% 	\includegraphics[scale=.35]{c1.png}\\
% 	This segment shows a reset followed by adding 170 and clearing.
% 	\includegraphics[scale=.35]{c2.png}\\
% 	This segment shows adding 254 and then subtracting 3 leaving 251 which can be seen at the beginning of the next segment.
% 	\includegraphics[scale=.35]{c3.png}\\
% 	This segment begins with a store of the answer previous to the address 255. Then a clear and we branch if not zero to 11001100 which can be seen in the next segment.
% 	\includegraphics[scale=.35]{c4.png}\\
% 	Finally we clear add 15 and branch if not zero to 11001100.
% \end{center}

	\newpage
\section{Significance} \vspace{-.7cm} \line(1,0){470}
	\paragraph{} 
		With the toy processor now built we can now execute very simple programs including tasks like add, subtract, clear, branch if not zero, and store. To improve its use, we can implement more features like ROM and RAM paths in order to execute more complex programs, and bring the whole implementation to the board. 

 \section{Comments/Suggestions}\vspace{-.7cm} \line(1,0){470}
 	\paragraph{} 
 		N.A.
		
\end{document}


